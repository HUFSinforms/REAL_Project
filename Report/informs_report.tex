%%%%%%%%%%%%
\documentclass[11pt]{article}
\usepackage{latexsym}
\usepackage{amsmath}
\usepackage[top=1in, bottom=1in, left=1in, right=1in]{geometry}


\parindent0pt
\parskip\bigskipamount

\begin{document}

\centerline{\textbf{2018 INFORMS O.R. {\&} Analytics Student Team Competition -- ENTRY FORM}}

\baselineskip16pt plus 1pt minus 1pt


\textbf{Entry Number: [}2018ORASTC252]

\textbf{Executive Summary (not to exceed 2 pages)}\\*


\textbf{Team Makeup {\&} Process}\\*

\textbf{Framing the Problem}\\*



\textbf{Data}\\*
The data from Principal consists of 3 parts.
- 데이터 구성 (risk index가 담겨있는 risk model 데이터, ~이 담겨있는 Timeseries data, 그리고 평가지표를 계산하기 위한 result data)
- 데이터에 대한 이해
- 모델에 적용시키기 위한 가공과정 
		-> Python의 Pandas Library를 이용하여 크게 3가지로 나뉜 데이터를 유동적으로 처리할 수 있도록 구성
		- >특히, date를 input으로 주면 각각의 데이터에서 해당 날짜의 데이터만 추출할 수 있도록 데이터 프레임화
		- >각 기간에 따라 (4weeks기준) 형성된 데이터 프레임에서 필요한 파라메터(alpha, beta ...)를 식별을 위해 Dictionary 형태로 저장(Key값: SEDOL)
- 데이터 Rescailing : 기본적으로 Python을 이용하였는데 값을 읽고 계산하는데 한계가 있음. 또한 optimization solver인 CPLEX를 이용하는데 값이 한계 수치가 있음. 따라서 값이 매우 작은 \omega 에 10000을 곱할 필요성 유. \omega 와 obj function에서 함께쓰이는 파라메터 \alpha 에도 10000을 곱함. (제약식도?)


\textbf{Methodology Approach {\&} Model Building}\\*

- 포뮬레이션을 그대로 CPLEX를 가지고 QCQP 풀었을 경우. (8)TE제약의 Non-convex를 어떻게 해결했는가? Bisearch 기법
	 -> non-tranditional constraints로 인해 optimal solution은 물론 feasible solution 또한 찾기 어려웠음 (간단한 실험결과,tab1) => 특히 cardinality 제약식이 bottle neck으로 작용

- 휴리스틱 알고리즘으로 진화 알고리즘 (ex.EDA)의 한계점. population의 수를 크게 두고 여러번 풀어 확률적 이론 등을 적용시키는 것이 이상적이나 문제를 푸는데 오랜시간 소요.

- DeepLearning 기법을 이용함 -> 세밀한 수치을 조정하기 위해서는 training 시간이 오래걸릴 뿐만 아니라 모든 제약식에 대해 feasible한 set을 찾기에는 무리가 있음. 


- 제시하는 모델의 전체적인 구조 제시 (Flow Chart)
	1) initial cardinality set 지정 : 
		- optimal solution을 도출하는 최적화 기법인 B&B algorithm을 이용했을때, tab1과 같이 cardinality 선택에서 어려움이 있으므로 이를 해소시키기 위한 방도. 
		- objective value(Return 대비 Risk) 가 낮아지는 좋은 cardinality set 을 찾아보자!
		- 3가지 방법(Random, Cplex, GAN)을 제시
		- 각 방법의 원리 설명

		
	2) 1)로부터 얻어진 cardinality set을 Cplex 수리모형에 적용 
	    - 실험 결과 및 정리 (Tab2)
	    
	3) 초기 해 개선 알고리즘
	
	4) Turnover 계산 후 다음 기간(date)에 반영


- Robustness고려



\textbf{Analytics Solution and Results}\\*




\begin{itemize}
\item \underline {This Entry Form}: Present your solution and results in this section, including completion of the Portfolio Performance Statistics chart below. You may supplement your analysis with additional charts, diagrams and/or other visualization; these supplements must be incorporated into this section of the Entry From.
\end{itemize}

\begin{center}
\textbf{Portfolio Performance Statistics}\vspace*{-14pt}
\end{center}
\begin{table}[htbp]
\def\arraystretch{1.4}
\begin{center}
\begin{tabular}{|l|c|c|}
\hline
\textbf{2007-01-01 to 2016-12-31}& 
\textbf{Portfolio}& 
\textbf{Benchmark} \\
\hline
Cumulative Return& 
{\%}& 
{\%} \\
\hline
Annualized Return& 
{\%}& 
{\%} \\
\hline
Annualized Excess Return& 
{\%}& 
-- \\
\hline
Annualized Tracking Error& 
{\%}& 
-- \\
\hline
Sharpe Ratio& 
& 
 \\
\hline
Information Ratio& 
& 
-- \\
\hline
\end{tabular}
\label{tab1}
\end{center}
\end{table}

\begin{itemize}
\item \underline {Results Template}: Populate and submit your numerical results using the Results Template. This template is provided as a separate file on the Competition download site. You can use either the Excel or CSV version.
\end{itemize}

\textbf{References}

\bgroup
\parskip0pt

Please follow guidelines in the \textit{Chicago Manual of Style,} 16$^{\text{th}}$ Edition. Here are examples: 

\begin{itemize}
\item[--] Journal article: Flynn J, Gartska SK (1990) A dynamic inventory model with periodic auditing. \textit{Oper. Res.} 38(6):1089--1103. 
\item[--] Book: Makridakis S, Wheelwright SC, McGee VE (1983) \textit{Forecasting: Methods and Applications}, 2nd ed. (John Wiley {\&} Sons, New York). 
\item[--] Edited Book: Martello S, Toth P (1979) The 0-1 knapsack problem. Christofides N, Mingozzi A, Sandi C, eds. \textit{Combinatorial Optimization} (John Wiley {\&} Sons, New York), 237--279.
\item[--] Online reference, fictional example: American Mathematical Institute (2005) Better predictors of geospatial variability. Retrieved June 14, 2005, \underline {www.mathematicsinstitute}.

\end{itemize}

\egroup

\end{document}
