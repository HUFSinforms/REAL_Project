%%%%%%%%%%%%
\documentclass[11pt]{article}
\usepackage{latexsym}
\usepackage{amsmath}
\usepackage[top=1in, bottom=1in, left=1in, right=1in]{geometry}


\parindent0pt
\parskip\bigskipamount

\begin{document}

\centerline{\textbf{2018 INFORMS O.R. {\&} Analytics Student Team Competition -- ENTRY FORM}}

\baselineskip16pt plus 1pt minus 1pt


\textbf{Entry Number: [}2018ORASTC252]

\textbf{Executive Summary (not to exceed 2 pages)}\\*
In this section, provide an executive summary for Principal executives. Your 
summary should briefly address your understanding of the business problem 
and how you approached solving it, covering not only the technical decisions 
and analysis but also the process your team used. The summary should also 
briefly describe your recommendations to Principal.

\textbf{Team Makeup {\&} Process}\\*
Without providing the names of individuals, describe the makeup of your team 
and the process your team used in working on the problem. This may include 
team members' background and experience, particularly as they may relate to 
the role each member played in the project. In addition, describe the 
process your team used, including elements such as how the work was 
allocated, how the team demonstrated the value of team members' ideas, how 
work from individuals was synthesized into the final analysis. You can also 
address any challenges and learnings from the team experience. 

\textbf{Framing the Problem}\\*
Describe your understanding of the business problem presented by Principal 
through the written problem statement, the webinar and other interaction 
with company executives, and their answers to questions posed by teams. Then 
explain how your team analyzed the requirements and determined the goal of 
the analysis. This may include determining which constraints to analyze, as 
well as defining a set of assumptions and key metrics of success.

\textbf{Data}\\*
Data for the problem was provided by Principal. If you used other data or 
sources, please define them here. Describe any work your team may have done 
with the data itself, such as rescaling, cleaning, identifying 
relationships, etc. If your analysis of the data helped to refine your 
understanding of the business and/or analytics problem, describe that here.

\textbf{Methodology Approach {\&} Model Building}\\*
In this section, describe the decision making process for selecting an 
analytics methodology(ies). What other methodologies did your team consider 
and what were the reasons for the final selection? You may want to include 
discussion and considerations -- such as the assumptions that were made, the 
scope and early considerations---to provide a useful framing for your 
selection. Then describe and document the chosen methodology and model in 
sufficient detail and clarity that it can be understood and evaluated. Your 
selection of software should also be addressed here. Given the 
multi-disciplinary aspect of the problem, background information may be 
useful to include or reference. 

\textbf{Analytics Solution and Results}\\*
In accordance with your methodology and model, present your analytics 
solution and results using \textbf{two forms:}

\begin{itemize}
\item \underline {This Entry Form}: Present your solution and results in this section, including completion of the Portfolio Performance Statistics chart below. You may supplement your analysis with additional charts, diagrams and/or other visualization; these supplements must be incorporated into this section of the Entry From.
\end{itemize}

\begin{center}
\textbf{Portfolio Performance Statistics}\vspace*{-14pt}
\end{center}
\begin{table}[htbp]
\def\arraystretch{1.4}
\begin{center}
\begin{tabular}{|l|c|c|}
\hline
\textbf{2007-01-01 to 2016-12-31}& 
\textbf{Portfolio}& 
\textbf{Benchmark} \\
\hline
Cumulative Return& 
{\%}& 
{\%} \\
\hline
Annualized Return& 
{\%}& 
{\%} \\
\hline
Annualized Excess Return& 
{\%}& 
-- \\
\hline
Annualized Tracking Error& 
{\%}& 
-- \\
\hline
Sharpe Ratio& 
& 
 \\
\hline
Information Ratio& 
& 
-- \\
\hline
\end{tabular}
\label{tab1}
\end{center}
\end{table}

\begin{itemize}
\item \underline {Results Template}: Populate and submit your numerical results using the Results Template. This template is provided as a separate file on the Competition download site. You can use either the Excel or CSV version.
\end{itemize}

\textbf{References}

\bgroup
\parskip0pt

Please follow guidelines in the \textit{Chicago Manual of Style,} 16$^{\text{th}}$ Edition. Here are examples: 

\begin{itemize}
\item[--] Journal article: Flynn J, Gartska SK (1990) A dynamic inventory model with periodic auditing. \textit{Oper. Res.} 38(6):1089--1103. 
\item[--] Book: Makridakis S, Wheelwright SC, McGee VE (1983) \textit{Forecasting: Methods and Applications}, 2nd ed. (John Wiley {\&} Sons, New York). 
\item[--] Edited Book: Martello S, Toth P (1979) The 0-1 knapsack problem. Christofides N, Mingozzi A, Sandi C, eds. \textit{Combinatorial Optimization} (John Wiley {\&} Sons, New York), 237--279.
\item[--] Online reference, fictional example: American Mathematical Institute (2005) Better predictors of geospatial variability. Retrieved June 14, 2005, \underline {www.mathematicsinstitute}.

\end{itemize}

\egroup

\end{document}
